% Options for packages loaded elsewhere
\PassOptionsToPackage{unicode}{hyperref}
\PassOptionsToPackage{hyphens}{url}
%
\documentclass[
]{article}
\usepackage{amsmath,amssymb}
\usepackage{lmodern}
\usepackage{iftex}
\ifPDFTeX
  \usepackage[T1]{fontenc}
  \usepackage[utf8]{inputenc}
  \usepackage{textcomp} % provide euro and other symbols
\else % if luatex or xetex
  \usepackage{unicode-math}
  \defaultfontfeatures{Scale=MatchLowercase}
  \defaultfontfeatures[\rmfamily]{Ligatures=TeX,Scale=1}
\fi
% Use upquote if available, for straight quotes in verbatim environments
\IfFileExists{upquote.sty}{\usepackage{upquote}}{}
\IfFileExists{microtype.sty}{% use microtype if available
  \usepackage[]{microtype}
  \UseMicrotypeSet[protrusion]{basicmath} % disable protrusion for tt fonts
}{}
\makeatletter
\@ifundefined{KOMAClassName}{% if non-KOMA class
  \IfFileExists{parskip.sty}{%
    \usepackage{parskip}
  }{% else
    \setlength{\parindent}{0pt}
    \setlength{\parskip}{6pt plus 2pt minus 1pt}}
}{% if KOMA class
  \KOMAoptions{parskip=half}}
\makeatother
\usepackage{xcolor}
\usepackage[margin=1in]{geometry}
\usepackage{graphicx}
\makeatletter
\def\maxwidth{\ifdim\Gin@nat@width>\linewidth\linewidth\else\Gin@nat@width\fi}
\def\maxheight{\ifdim\Gin@nat@height>\textheight\textheight\else\Gin@nat@height\fi}
\makeatother
% Scale images if necessary, so that they will not overflow the page
% margins by default, and it is still possible to overwrite the defaults
% using explicit options in \includegraphics[width, height, ...]{}
\setkeys{Gin}{width=\maxwidth,height=\maxheight,keepaspectratio}
% Set default figure placement to htbp
\makeatletter
\def\fps@figure{htbp}
\makeatother
\setlength{\emergencystretch}{3em} % prevent overfull lines
\providecommand{\tightlist}{%
  \setlength{\itemsep}{0pt}\setlength{\parskip}{0pt}}
\setcounter{secnumdepth}{-\maxdimen} % remove section numbering
\ifLuaTeX
  \usepackage{selnolig}  % disable illegal ligatures
\fi
\usepackage[]{natbib}
\bibliographystyle{plainnat}
\IfFileExists{bookmark.sty}{\usepackage{bookmark}}{\usepackage{hyperref}}
\IfFileExists{xurl.sty}{\usepackage{xurl}}{} % add URL line breaks if available
\urlstyle{same} % disable monospaced font for URLs
\hypersetup{
  pdftitle={Introducción al análisis de datos con R},
  pdfauthor={Ejercicios - Sesión 3},
  hidelinks,
  pdfcreator={LaTeX via pandoc}}

\title{Introducción al análisis de datos con R}
\usepackage{etoolbox}
\makeatletter
\providecommand{\subtitle}[1]{% add subtitle to \maketitle
  \apptocmd{\@title}{\par {\large #1 \par}}{}{}
}
\makeatother
\subtitle{Introducción a la visualización de datos y flujos de trabajo}
\author{Ejercicios - Sesión 3}
\date{5 - 9 junio de 2023}

\begin{document}
\maketitle

\hypertarget{introducciuxf3n}{%
\subsection{Introducción}\label{introducciuxf3n}}

En este ejercicio vamos a trabajar con datos procedentes del proyecto
``Atlas de Oportunidades''. Este proyecto constituye el primer análisis
de movilidad social en España basado en datos de rentas de padres e
hijos, proporcionados por la Agencia Estatal de Administración
Tributaria.

El objetivo de este ejercicio es visualizar la relación entre la
influencia de la renta de los padres y la renta futura de los hijos,
teniendo en cuenta diferentes características como el sexo y el
territorio de residencia.

\hypertarget{primeros-pasos}{%
\subsection{Primeros pasos}\label{primeros-pasos}}

\begin{itemize}
\tightlist
\item
  Carga la base de datos ``tabla\_nacional.csv'' ubicada en la carpeta
  ``datatset\_atlas\_oportunidad'' y almacénala en un objeto llamado
  ``df''.
\end{itemize}

\hypertarget{representaciuxf3n-gruxe1fico-de-los-datos-dplyr-ggplot}{%
\subsection{\texorpdfstring{Representación gráfico de los datos:
\texttt{dplyr} +
\texttt{ggplot}}{Representación gráfico de los datos: dplyr + ggplot}}\label{representaciuxf3n-gruxe1fico-de-los-datos-dplyr-ggplot}}

\begin{itemize}
\item
  Filtra la base de datos para obtener un data frame llamado ``dftotal''
  que cumpla con los siguientes filtros: sexo = ``total'', tipo\_renta =
  ``individual'' y promedio = ``mediana''. Una vez filtrada la base de
  datos, crea un gráfico de dispersión de puntos utilizando
  \texttt{geom\_point()}. En el eje X, representa la variable
  ``centil\_padres'', y en el eje Y, la variable
  ``centil\_hijo\_loess''.
\item
  Sustituye los puntos por una línea utilizando la función
  \texttt{geom\_line()}.
\item
  En una sociedad donde la renta de los padres no influye en la renta
  futura de los hijos, la representación gráfica sería una línea recta
  en el valor 50 del eje Y. Utilizando \texttt{geom\_hline()} , añade
  una línea recta que intersecte el gráfico en el percentil 50 del eje
  Y. Además, cambia el color de la línea usando
  \texttt{color="\#6D92CE"} y el tamaño usando \texttt{size=1}.
\item
  Personaliza el gráfico agregando un título sugerente y etiquetando las
  variables correspondientes en los ejes X e Y. Además, incluye la
  fuente de los datos utilizando la función \texttt{labs()}.
\item
  A continuación, vamos a representar la desigualdad de oportunidades
  para hombres y mujeres. Crea un nuevo data frame llamado ``df\_sexo''
  donde filtres las siguientes columnas por las siguientes categorías:
  tipo\_renta = ``individual'' y promedio = ``mediana''. Una vez que
  tengas los datos preparados, crea un gráfico de líneas utilizando
  \texttt{geom\_line()} e incluye los valores para la variable ``sexo''.
  Utiliza diferentes colores para cada categoría.
\item
  Cambia el color de las líneas de tendencia. Utiliza color naranja para
  los hombres, color morado para las mujeres y color negro para el
  total. Puedes hacerlo mediante la función
  \texttt{scale\_color\_manual()}.
\item
  Modifica el eje Y para que vaya desde el valor 20 hasta el valor 100.
\item
  Ahora, vamos a analizar si la desigualdad de oportunidades difiere
  según la Comunidad Autónoma de residencia.

  \begin{itemize}
  \tightlist
  \item
    Primero, carga el data frame ``tabla\_ccaa'' y almacénalo en un
    nuevo objeto llamado ``dfccaa''.
  \item
    Segundo, crea un gráfico donde representes la relación entre la
    renta de padres e hijos en cada Comunidad Autónoma (utilizando
    \texttt{facet\_wrap()}) para hombres y mujeres.
  \item
    Tercero, personaliza el gráfico con el fin de mejorar su
    comprensión. Por ejemplo, cambia el color para hombres (naranja) y
    mujeres (morado), cambia el tema a \texttt{theme\_bw()}, y añade
    títulos y fuente.
  \end{itemize}
\item
  Guarda el gráfico usando la función \texttt{ggsave()}.
\end{itemize}

\end{document}
