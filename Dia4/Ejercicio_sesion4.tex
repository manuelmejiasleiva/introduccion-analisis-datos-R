% Options for packages loaded elsewhere
\PassOptionsToPackage{unicode}{hyperref}
\PassOptionsToPackage{hyphens}{url}
%
\documentclass[
]{article}
\usepackage{amsmath,amssymb}
\usepackage{lmodern}
\usepackage{iftex}
\ifPDFTeX
  \usepackage[T1]{fontenc}
  \usepackage[utf8]{inputenc}
  \usepackage{textcomp} % provide euro and other symbols
\else % if luatex or xetex
  \usepackage{unicode-math}
  \defaultfontfeatures{Scale=MatchLowercase}
  \defaultfontfeatures[\rmfamily]{Ligatures=TeX,Scale=1}
\fi
% Use upquote if available, for straight quotes in verbatim environments
\IfFileExists{upquote.sty}{\usepackage{upquote}}{}
\IfFileExists{microtype.sty}{% use microtype if available
  \usepackage[]{microtype}
  \UseMicrotypeSet[protrusion]{basicmath} % disable protrusion for tt fonts
}{}
\makeatletter
\@ifundefined{KOMAClassName}{% if non-KOMA class
  \IfFileExists{parskip.sty}{%
    \usepackage{parskip}
  }{% else
    \setlength{\parindent}{0pt}
    \setlength{\parskip}{6pt plus 2pt minus 1pt}}
}{% if KOMA class
  \KOMAoptions{parskip=half}}
\makeatother
\usepackage{xcolor}
\usepackage[margin=1in]{geometry}
\usepackage{graphicx}
\makeatletter
\def\maxwidth{\ifdim\Gin@nat@width>\linewidth\linewidth\else\Gin@nat@width\fi}
\def\maxheight{\ifdim\Gin@nat@height>\textheight\textheight\else\Gin@nat@height\fi}
\makeatother
% Scale images if necessary, so that they will not overflow the page
% margins by default, and it is still possible to overwrite the defaults
% using explicit options in \includegraphics[width, height, ...]{}
\setkeys{Gin}{width=\maxwidth,height=\maxheight,keepaspectratio}
% Set default figure placement to htbp
\makeatletter
\def\fps@figure{htbp}
\makeatother
\setlength{\emergencystretch}{3em} % prevent overfull lines
\providecommand{\tightlist}{%
  \setlength{\itemsep}{0pt}\setlength{\parskip}{0pt}}
\setcounter{secnumdepth}{-\maxdimen} % remove section numbering
\ifLuaTeX
  \usepackage{selnolig}  % disable illegal ligatures
\fi
\usepackage[]{natbib}
\bibliographystyle{plainnat}
\IfFileExists{bookmark.sty}{\usepackage{bookmark}}{\usepackage{hyperref}}
\IfFileExists{xurl.sty}{\usepackage{xurl}}{} % add URL line breaks if available
\urlstyle{same} % disable monospaced font for URLs
\hypersetup{
  pdftitle={Introducción al análisis de datos con R},
  pdfauthor={Ejercicios - Sesión 4},
  hidelinks,
  pdfcreator={LaTeX via pandoc}}

\title{Introducción al análisis de datos con R}
\usepackage{etoolbox}
\makeatletter
\providecommand{\subtitle}[1]{% add subtitle to \maketitle
  \apptocmd{\@title}{\par {\large #1 \par}}{}{}
}
\makeatother
\subtitle{Introducción al análisis de regresión con R}
\author{Ejercicios - Sesión 4}
\date{5 - 9 junio de 2023}

\begin{document}
\maketitle

\hypertarget{ejercicio-1-regresiuxf3n-lineal-simple}{%
\section{Ejercicio 1: regresión lineal
simple}\label{ejercicio-1-regresiuxf3n-lineal-simple}}

Se cuenta con un conjunto de datos que incluyen información sobre
indicadores educativos y la renta media de cada barrio en la ciudad de
Madrid. El objetivo es analizar si existe una relación entre la
titularidad de los colegios (público vs privado-concertado) y la renta
media del barrio en los indicadores de rendimiento educativo.

Información sobre la base de datos:
\url{https://github.com/marespadafor/panelmadrid/blob/main/panel_information.md}

Pasos a seguir:

\begin{itemize}
\item
  Cargar los datos desde el archivo \texttt{madrid\_schools.dta} y
  explorar su estructura y distribución. Para esto se pueden utilizar
  las funciones \texttt{read.dta()} y \texttt{summary()} en R.
\item
  Crear un gráfico de dispersión para visualizar la relación entre las
  variables renta media del barrio y el porcentaje de alumnos que
  completan la ESO en cada colegio. Para ello, se puede utilizar la
  función \texttt{ggplot()} del paquete ggplot2.
\item
  Ajustar un modelo de regresión lineal simple para predecir el
  porcentaje de alumnos que completan la ESO en función de la renta
  media del barrio donde se ubica cada colegio. Para esto se puede
  utilizar la función \texttt{lm()} en R.
\item
  Evaluar el modelo de regresión lineal simple utilizando el coeficiente
  de determinación (R²). Para esto, se puede utilizar la función
  \texttt{summary()} en R.
\item
  Realizar una predicción del porcentaje de alumnos que completan la ESO
  en función de la renta media de los barrios. Para ello, se puede
  utilizar la función \texttt{ggeffect()} en R.
\end{itemize}

\hypertarget{ejercicio-2-regresiuxf3n-lineal-multivariante}{%
\section{Ejercicio 2: regresión lineal
multivariante}\label{ejercicio-2-regresiuxf3n-lineal-multivariante}}

Basándonos en el mismo supuesto que en el ejercicio anterior, vamos a
demostrar cómo la titularidad del colegio se relaciona con el porcentaje
de estudiantes que completan la Educación Secundaria Obligatoria (ESO),
en función de la renta media del barrio donde se encuentra ubicado el
colegio.

\begin{itemize}
\item
  Ajustar un modelo de regresión lineal para predecir el porcentaje de
  alumnos que completan la ESO en función de la renta media del barrio
  donde se ubica cada colegio, añadiendo un término de interacción entre
  la renta del barrio y la titularidad del colegio. Esto permitirá
  evaluar si la relación entre la renta del barrio y el porcentaje de
  alumnos que completan la ESO difiere según la titularidad del colegio.
\item
  Realizar una predicción del porcentaje de alumnos que completan la ESO
  según la titularidad del centro en función de la renta media de los
  barrios. Para ello, se puede utilizar la función ggeffect() en R, que
  permitirá visualizar los efectos de la renta del barrio y la
  titularidad del colegio en la predicción del porcentaje de alumnos que
  completan la ESO. Finalmente, se deberá interpretar el gráfico para
  entender cómo se relacionan las variables
\end{itemize}

\end{document}
