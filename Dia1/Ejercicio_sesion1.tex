% Options for packages loaded elsewhere
\PassOptionsToPackage{unicode}{hyperref}
\PassOptionsToPackage{hyphens}{url}
%
\documentclass[
]{article}
\usepackage{amsmath,amssymb}
\usepackage{lmodern}
\usepackage{iftex}
\ifPDFTeX
  \usepackage[T1]{fontenc}
  \usepackage[utf8]{inputenc}
  \usepackage{textcomp} % provide euro and other symbols
\else % if luatex or xetex
  \usepackage{unicode-math}
  \defaultfontfeatures{Scale=MatchLowercase}
  \defaultfontfeatures[\rmfamily]{Ligatures=TeX,Scale=1}
\fi
% Use upquote if available, for straight quotes in verbatim environments
\IfFileExists{upquote.sty}{\usepackage{upquote}}{}
\IfFileExists{microtype.sty}{% use microtype if available
  \usepackage[]{microtype}
  \UseMicrotypeSet[protrusion]{basicmath} % disable protrusion for tt fonts
}{}
\makeatletter
\@ifundefined{KOMAClassName}{% if non-KOMA class
  \IfFileExists{parskip.sty}{%
    \usepackage{parskip}
  }{% else
    \setlength{\parindent}{0pt}
    \setlength{\parskip}{6pt plus 2pt minus 1pt}}
}{% if KOMA class
  \KOMAoptions{parskip=half}}
\makeatother
\usepackage{xcolor}
\usepackage[margin=1in]{geometry}
\usepackage{graphicx}
\makeatletter
\def\maxwidth{\ifdim\Gin@nat@width>\linewidth\linewidth\else\Gin@nat@width\fi}
\def\maxheight{\ifdim\Gin@nat@height>\textheight\textheight\else\Gin@nat@height\fi}
\makeatother
% Scale images if necessary, so that they will not overflow the page
% margins by default, and it is still possible to overwrite the defaults
% using explicit options in \includegraphics[width, height, ...]{}
\setkeys{Gin}{width=\maxwidth,height=\maxheight,keepaspectratio}
% Set default figure placement to htbp
\makeatletter
\def\fps@figure{htbp}
\makeatother
\setlength{\emergencystretch}{3em} % prevent overfull lines
\providecommand{\tightlist}{%
  \setlength{\itemsep}{0pt}\setlength{\parskip}{0pt}}
\setcounter{secnumdepth}{-\maxdimen} % remove section numbering
\ifLuaTeX
  \usepackage{selnolig}  % disable illegal ligatures
\fi
\usepackage[]{natbib}
\bibliographystyle{plainnat}
\IfFileExists{bookmark.sty}{\usepackage{bookmark}}{\usepackage{hyperref}}
\IfFileExists{xurl.sty}{\usepackage{xurl}}{} % add URL line breaks if available
\urlstyle{same} % disable monospaced font for URLs
\hypersetup{
  pdftitle={Introducción al análisis de datos con R},
  pdfauthor={Ejercicios - Sesión 1},
  hidelinks,
  pdfcreator={LaTeX via pandoc}}

\title{Introducción al análisis de datos con R}
\usepackage{etoolbox}
\makeatletter
\providecommand{\subtitle}[1]{% add subtitle to \maketitle
  \apptocmd{\@title}{\par {\large #1 \par}}{}{}
}
\makeatother
\subtitle{Conceptos básicos de R y RStudio}
\author{Ejercicios - Sesión 1}
\date{5 - 9 junio de 2023}

\begin{document}
\maketitle

\hypertarget{introducciuxf3n}{%
\subsection{1. Introducción}\label{introducciuxf3n}}

En este ejercicio vamos a utilizar datos transversales de 675 niños de
14 años nacidos entre 1980 y 1988. La muestra procede del Panel
Socioeconómico Alemán (GSOEP) de los años 1994 a 2002 para investigar
los determinantes de la elección de centro de enseñanza secundaria. En
el sistema educativo alemán, los niveles de Gymnasium, Hauptschule y
Realschule corresponden a diferentes tipos de escuelas secundarias con
enfoques educativos y académicos distintos:

\begin{itemize}
\item
  La Hauptschule es una escuela secundaria de nivel básico y se
  considera el nivel educativo más bajo en el sistema alemán.
\item
  La Realschule es una escuela secundaria de nivel intermedio.
\item
  El Gymnasium es una escuela secundaria de nivel más alto y
  académicamente orientada (los estudiantes se preparan para ingresar a
  la universidad).
\end{itemize}

La documentación sobre las variables que contiene esta base de datos se
encuentra disponible en el siguiente enlace:
\url{https://vincentarelbundock.github.io/Rdatasets/doc/AER/GSOEP9402.html}

\hypertarget{primeros-pasos}{%
\subsection{2. Primeros pasos}\label{primeros-pasos}}

En los ejercicios de la primera sesión vamos a trabajar sobre cómo
realizar un primer acercamiento exploratorio a los datos con R. Antes de
empezar:

\begin{itemize}
\item
  2.1. Abre el script \texttt{Ejercicio\_sesion1} y limpia el espacio de
  datos ejecutando \texttt{rm(list\ =\ ls())}. Este comando eliminará
  todos los datos (objetos) del espacio de trabajo, evitando posibles
  confusiones.
\item
  2.2. Carga los paquetes que necesitas para realizar el ejercicio,
  ejecutando las líneas de \texttt{library()}. En caso de que alguno de
  ellos no esté instalado, instalalo utilizando
  \texttt{install.packages("package")}. Fudamentalmente, en este
  ejercicio se utilizará la librería \textbf{tidyverse} y
  \textbf{janitor}.
\item
  2.3. Establece el directorio de trabajo donde se encuentran los datos.
  Puedes hacerlo usando \texttt{setwd()} o desde \texttt{Files}.
\item
  2.4. Carga el archivo de datos en formato CSV utilizando
  \texttt{read\_csv()} y asígnalo a un objeto llamado \textbf{data}.
\end{itemize}

\hypertarget{observando-la-estructura-de-los-datos}{%
\subsection{3. Observando la estructura de los
datos}\label{observando-la-estructura-de-los-datos}}

Explora la base de datos que acabas de cargar:

\begin{itemize}
\tightlist
\item
  3.1. Muestra las primeras 6 filas utilizando \texttt{head()}.
\item
  3.2. Explora la estructura de los datos utilizando \texttt{glimpse()}.
  ¿Cuántas filas tiene? ¿Cuántas columnas tiene? ¿De qué tipo son las
  variables?
\item
  3.3. Verifica si existen valores perdidos en alguna columna utilizando
  \texttt{colSums(is.na())}.
\end{itemize}

\hypertarget{anuxe1lisis-exploratorio-de-los-datos}{%
\subsection{4. Análisis exploratorio de los
datos}\label{anuxe1lisis-exploratorio-de-los-datos}}

\begin{itemize}
\tightlist
\item
  4.1. Calcula la media de años de educación que tienen las madres
  utilizando \texttt{mean()}.
\item
  4.2. Calcula la mediana de los ingresos del hogar utilizando
  \texttt{median()}.
\item
  4.3. Calcula la media de años de educación de la madre para cada
  itinerario educativo cursado por los alumnos. Utiliza
  \texttt{aggregate()}.
\item
  4.4. Calcula la media de ingresos del hogar para cada itinerario
  educativo cursado por los alumnos. Utiliza \texttt{aggregate()}.
\item
  4.5. Realiza un análisis para mostrar, en términos porcentuales, si la
  elección de los itinerarios educativos (school) difiere según el
  estado civil de la madre (marital). Utiliza \texttt{tabyl()}.
\end{itemize}

\end{document}
